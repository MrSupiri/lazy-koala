\newpage
{\let\clearpage\relax \chapter{Research Methodology}}

\begin{longtable}{|p{4cm}|p{10cm}|}
\hline
  \textbf{Research Philosophy} & 
  Mainly, there are four research philosophies, Pragmatism, positivism, realism, and interpretivism. It explains the belief and the research is done. After doing an in-depth study about research philosophies, the author decided on following \textbf{Pragmatism} as the research philosophy because the author believes there is no one way to solve the problem this research is tried to address and the goal of this research is to solve a practical problem faced by \acp{sres}. (\cite{1Philoso75:online}, \cite{Pragmati87:online})
  \\ \hline
  
  \textbf{Research Approach} & 
  Although the inspiration for the research came from an observation of the real world. The author is using \textbf{deductive reasoning} to approach the problem. After the problem was identified the author looked for existing work found few theories on the domain. Then the author found few flaws in these methods thought of a way to address them with different approaches. At the end of the research other hopes to implement these new approaches and observe their outcome.
  \\ \hline
  
  \textbf{Research Strategy} & 
  The research strategy will be used to answer the research questions. In this project, the author will use \textbf{Experimenting, interviews, and surveys} to provide answers to research questions.
  \\ \hline
  
  \textbf{Research Choice} & 
  During this research project, the author is planning to build a very generalized solution to predict anomalies.  So to achieve this a \textbf{quantitative} dataset will be used and since only one data collection method will be used this research is done under the\textbf{ Mono method}.
  \\ \hline
  
  Time zone & 
  This project needs to be completed within 8 months, so a \textbf{cross-sectional} time horizon will be used to collect data to complete the project.
  \\ \hline
\end{longtable}