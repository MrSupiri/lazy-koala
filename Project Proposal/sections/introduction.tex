\section{Introduction}

There was a big shift towards cloud computing in recent years due to its scalability and  ease of use. With this change, a new programming paradigm called cloud-native was born. Cloud-native applications are often developed as a set of stand-alone microservices \citep{dragoni2017microservices} yet could depend on each other to provide a unified experience. This helps different teams to work on different services which increases the development velocity. This works well for medium to large companies but over time when this mesh of services could become very complicated to a point where it's very difficult for one person to understand the whole system.

When the system consists of 1000s of individual services talking and depending on each other, the network layer of that system becomes chaotic \citep{Introduc54:online} and failure in a single point can create a ripple effect across the system. When something like that happens it's really difficult to zero in on the exact point of failure quickly. In this research, The author is planning to introduce a system that monitors all the services and detects if the service is acting out of shape and if so find the root cause of it by evaluating it on a dependency tree.