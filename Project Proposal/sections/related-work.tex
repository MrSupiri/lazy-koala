\section{Related Work}

As the large-scale migration towards the cloud and microservices started fairly recently the problem this research is trying to solve mostly affects large-scale enterprises there ain't a lot of published research on this domain. All the work done towards uncovering the root cause of failures by large co-operations either kept their finds for internal use to sell it as \ac{saas} product. 

One of the best implementations found on root cause analysis is from Datadog. They created a platform called watchdog \citep{Watchdog76:online} which monitors the entire system for anomalies and failures in the background. When a failure happens it tries to pull all the relevant stack traces and monitoring data to a single view so the developer can diagnose the problem easily. The problem with this solution is even though it was announced all the way back in July 2018, all that is available is currently in private beta which not everyone has access to.
\\
All the currently published work on microservices monitoring can be classified into 2 categories
\begin{enumerate}
\item Anomaly detection
\item Root cause identification
\end{enumerate}

\subsection{Anomaly detection}

Anomaly detection in time series is a field of its own. So in this case we will be forcing papers that are specialized in the cloud computing domain.

One of the earliest attempts on detecting anomalies in microservices was \cite{du2018anomaly}. In this authors tried using 4 different machine learning techniques to detect performance anomalies. To do this say used a simulated system and various fault injection mechanisms to create the dataset. In the end, they concluded K Nearest Neighbors classifier gives the most accurate classifications while Support vector machines have the worse.

A common way to detect anomalies in time series is using an autoencoder to reconstruct a given time series. After training the model should be able to come up with the generalized function about the given time series and it will be able to recreate any input sequence accurately. But when there is an anomaly in the input sequence models output will be vastly different from the input. We can use this reconstruction loss as a metric to uncover anomalies within the system. In \cite{kumarage2018anomaly} authors used the method to detect anomalies in distributed systems. In a continuation of their work \cite{kumarage2019generative} they tried doing the same thing by using a \ac{gan} but in the end, they concluded even though it showed a tendency towards better performance when the dataset gets bigger, with the dataset they had autoencoders perform well overall.

Ever since DeepMind came up with wavenet which used a CNN to generate audio samples \citep{oord2016wavenet} researchers uncovering other potential use cases other than image-related tasks. One of those use cases was as CNN excels at pattern recognition, encoding time series data set into image-like data structures and use a CNN to identify abnormal patterns in it. On \cite{kim2018encoding} authors tried to using a novel technique to raw encode data into a pixel-like structure and found it could outperform the existing methods to detect anomalies in computer networks.

\subsection{Root cause identification}

Predicting the exact root cause of failure just using a standard machine learning model is a pretty difficult task since prediction space is not finite. In 2017 a team from Google X tried using the Bayesian Network to model the relationship between the state of the system and its effect on failures \citep{chigurupati2017root}. Using it they were able to accurately predict all the possible root causes of a hardware failure in certain systems but this model required to predefine all the possible error modes by domain experts which isn't really possible in a constantly evolving distributed system. There were similar attempts \cite{gonzalez2017root} to use machine learning to generate knowledge graphs on historical data and help developers come up with reasoning to failures although this eliminated a need for a domain expert, this also can't react to unseen errors.

In a distributed system it's hard to spot real anomalies just by looking at monitoring data, but when there are huge spikes in response latencies or error rates it's a good indicator something must be wrong. So \cite{samir2019dla} used a \ac{hhmm} to uncover the possible affected services from changes in response time or error rates in one service and using that data to uncover the root cause of the issue. All of the papers discussed above have one problem in common they all assume the entire system is static but in reality, these services changes over time either with increased demand or new future implementations. To address this, \cite{wu2020microrca} developed a service that monitors all the running applications and their vitals. This also constructs an attributed graph that represents how each service interacts with the other. When the monitoring system detects an anomaly MicroRCA weight that graph with response time changes and tries to find the epicenter of the anomaly. The main problem with both of these approaches have is authors rely solely on slow response time as an indication of an anomaly but several other factors could course anomalous behaviors without changes in response times.
