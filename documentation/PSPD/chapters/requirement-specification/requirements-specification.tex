\section{Requirements Specifications}

Since this project touched the deep ends of both \ac{sre} and data science expectations and prioritizes of the system must be managed to achieve a reliable and functioning system at end of the deadline. So to achieve this MoSCoW prioritization system was used.

\begin{longtable}{|p{25mm}|p{128mm}|}
\hline
    \textbf{Priority Level} &
    \textbf{Description} \\ \hline
    
    Critical  &
    Requirements which need to be met in order to have minimum viable state. \\ \hline

    Important &
    Requirements that need to be completed to have a usable product. \\ \hline

    Luxury &
    Nice to have requirements that improve the quality of life of the system. \\ \hline
\caption{Requirement oriorities (self-composed)}
\end{longtable}

\begin{longtable}{|p{35mm}|p{118mm}|}
\hline
    \textbf{Use Case ID} & \textbf{Use Case Name }              \\ \hline
    UC-01                & Deploy Lazy Koala                    \\ \hline
    UC-02                & Update Configuration                 \\ \hline
    UC-03                & Purge Lazy Koala                     \\ \hline
    UC-04                & Check for Root Courses               \\ \hline
    UC-05                & Generate Report                      \\ \hline
    UC-06                & Read from the database               \\ \hline
    UC-07                & Extract telemetry (Every 5 second)   \\ \hline
    % UC-08                & Update Network topology              \\ \hline
    UC-08                & Check for Anomalies (Every 1 minute) \\ \hline
    UC-09                & Write to the database                \\ \hline
    UC-10                & Reconcile on modified resources      \\ \hline
    UC-11                & Update cluster state                 \\ \hline
\caption{Use cases of the system (self-composed)}
\end{longtable}

% \newpage

\newcommand{\functionalRequirement}[5]{
    #1 &
    % \makecell[{{p{109mm}}}]{\textbf{#2}\\#3} &
    \textbf{#2} \newline #3 &
    #4 &
    #5  \\ \hline
}

\subsection{Functional Requirements}

\begin{longtable}{|p{9mm}|p{107mm}|p{16mm}|p{13mm}|}
\hline
    \textbf{ID} &
    \textbf{Requirement and Description} &
    \textbf{Priority Level} &
    \textbf{Use case} \\ \hline


    \functionalRequirement
    {FR01}
    {Users should be able to deploy the Lazy Koala to an existing Kubernetes cluster.}
    {Lazy koala should work on any Linux-based Kubernetes cluster with version 1.22 or higher without any additional configuration from the user's end.}
    {Critical}
    {UC-01}
    
    \functionalRequirement
    {FR02}
    {Users should be able to remove Lazy Koala completely from the cluster.}
    {Once uninstalled all the provisioned resources should be cleaned up by the \ac{lazy-koala-operator} itself. }
    {Important}
    {UC-03}
    
        
    \functionalRequirement
    {FR03}
    {Users should be able to specify which services need to be monitored.}
    {System should allow the user to exclude some services getting tracked.}
    {Critical}
    {UC-02}
    
        
    \functionalRequirement
    {FR04}
    {Users should be able to see the services monitored by Lazy Koala.}
    {System should be transparent to the user about monitored and unmonitored service.}
    {Luxury}
    {UC-05}
    
        
    \functionalRequirement
    {FR05}
    {Lazy Koala should deploy an instance of \ac{gazer} to every node in the cluster.}
    {In the Kubernetes cluster every node has a separate instance of the Linux kernel. So for every instance of Linux kernel, an instance \ac{gazer} must be present to ensure all the relevant data is captured.}
    {Critical}
    {UC-01, UC-10}
    
        
    \functionalRequirement
    {FR06}
    {\ac{gazer} should intersect all “inet\_sock\_set\_state” kernel calls and export the relevant data to Prometheus.}
    {Whenever a userspace application makes a TCP call this kernel method is called to communicate with the network interface. Inspecting the data structures of this will allow us to extract a lot of information about each TCP calls.}
    {Critical}
    {UC-07, UC-09}
    
        
    \functionalRequirement
    {FR07}
    {\ac{gazer} periodically polls the size of “sk\_ack\_backlog” for interested ports to export the relevant data to Prometheus.}
    {This kernel data structure holds the TCP connections that are left to be acknowledged. Knowing the size of this queue will help to understand the efficiency of each service.}
    {Important}
    {UC-07, UC-09}
    
        
    \functionalRequirement
    {FR08}
    {\ac{gazer} should poll Kubernetes metric server periodically and export the relevant data to Prometheus.}
    {Sudden changes in CPU and Memory usage will be a good indication for an anomaly. So using those to be processed later will be wise.}
    {Critical}
    {UC-07, UC-09}
    
        
    \functionalRequirement
    {FR09}
    {Lazy Koala should periodically check for changes in monitored services and update the \ac{gazer} ConfigMap.}
    {Kubelet is watching over all the services on the system and restarts them if they become unhealthy. With that, the IP address of that service is gonna change and Lazy Koala is responsible to let the \ac{gazer} know of such changes.}
    {Important}
    {UC-10, UC-11}
    
        
    \functionalRequirement
    {FR10}
    {\ac{gazer} should react to config updates in realtime.}
    {When a lazy koala pushes a new config, \ac{gazer} should look for the new IPs without requiring a complete reset. Which is time consuming and expensive.}
    {Luxury}
    {UC-10, UC-11}
    
        
    \functionalRequirement
    {FR11}
    {Lazy Koala should provision an instance of \ac{sherlock} for each of the monitored services.}
    {\ac{sherlock} is responsible for analyzing all the metric data exported from \ac{gazer}.}
    {Critical}
    {UC-10, UC-11}
    
        
    \functionalRequirement
    {FR12}
    {\ac{sherlock} should periodically calculate the anomaly score for each of the monitored services and export it to Prometheus.}
    {Anomaly score is used by the UI to understand the spread of an anomaly.}
    {Critical}
    {UC-08, UC-09}
    
        
    \functionalRequirement
    {FR13}
    {Lazy Koala should have a Web UI visualized service topology.}
    {UI should help users to visualize the spread of an anomaly throughout all of the monitored service.}
    {Important}
    {UC-04, UC-05, UC-06}
    
        
    \functionalRequirement
    {FR14}
    {Lazy Koala should add a finalizer for each of the provisioned resources.}
    {Finalizers ensure the parent of a resource won’t be deleted before all the children are cleaned up which avoids leaving the cluster with orphaned resources that won’t be cleaned up.}
    {Important}
    {UC-03}
    
    
    

\caption{Functional requirements (self-composed)}
\end{longtable}

\subsection{Non-Functional Requirements}

\begin{longtable}{|p{13mm}|p{89mm}|p{26mm}|p{18mm}|}
\hline
    \textbf{ID} &
    \textbf{Description} &
    \textbf{Specification} &
    \textbf{Priority Level} \\ \hline
    
    NFR1 &
    Lazy Koala should follow Principle of Least Privilege when accessing Kubernetes APIs. &
    Security &
    Critical \\ \hline
    
    NFR2 &
    Systems should have fragment architecture so each component can be individually scaled to save resources. &
    Scalability &
    Critical \\ \hline
    
    NFR3 &
    Each component should work individually such that users can install part of the system they are interested in. &
    Usability &
    Desirable \\ \hline
    
    NFR4 &
    \ac{gazer} should be limited to using only 100 mCPUs and 80MB of memory. &
    Performance &
    Important \\ \hline
    
    NFR5 &
    \ac{sherlock} should be limited to using only 100 mCPUs and 100MB of memory. &
    Performance &
    Desirable \\ \hline
    
    NFR6 &
    Lazy Koala should be packaged as a Helm Chart for ease of use. &
    Usability &
    Critical \\ \hline
    
    NFR7 &
    Reconstruction error of \ac{sherlock} should be under 0.1\% &
    Performance &
    Desirable \\ \hline
    
    NFR8 &
    Lazy Koala’s reconciliation loop should use exponential backoff when an error occurred while reconciling a config change. &
    Reliability &
    Critical \\ \hline
    
    NFR9 &
    Follow Coding best practices and rely on linter. &
    Maintainability &
    Desirable \\ \hline
    
    NFR10 &
    The project should be backed by an automated CI/CD tool to test and build each component with each release. &
    Maintainability &
    Desirable \\ \hline

\caption{Non-Functional requirements (self-composed)}
\end{longtable}