\section{Requirements Elicitation Methodologies}

When developing a software project, one of the very first steps that need to be done is requirements engineering. Without following this process, it is difficult to come up with a product that users actually want to use. In this section, the author will describe the techniques he has used to gather requirements with their results.

\begin{longtable}{|p{160mm}|}
\hline
\textbf{Literature Review} \\ \hline
A Literature review is the fundamental building block of any research project. It helps to understand existing systems and how they work, and also Issues and gaps in those established systems. Since this research project has 3 sub-components to make up the full system a literature review was done on existing instrumentation, anomaly detection, and root course localization systems. \\ \hline

\textbf{Interviews} \\ \hline
The target audience for this project will be mostly reliability engineers. Having a one-on-one interview with them and discussing my project idea and the implementation path uncovered some overlooked use cases and possible improvements to the current implementation. \\ \hline


\textbf{Self-evaluation} \\ \hline
Since the initial idea for the project came from a difficult author faced maintaining a distributed system during the industrial placement period. The author was able to carry out several self-evaluations during the course of the project and realigned the project scope with the original issue so the project is always on track. \\ \hline

\textbf{Brainstorming} \\ \hline
As mentioned above since this project originated from a practical problem the author faced. The author was able to use his own experience and self brainstorm some key requirements of the project that he would personally like to be included. \\ \hline

\textbf{Prototyping} \\ \hline
As the system is getting built, requirements get added and removed due to some requirements getting too complex to build or some additional requirements need to be met to have the core functionality working. This also gives an opportunity to share the current progress with a subset of the target audience and get their feedback and improve upon them. \\ \hline

\caption{Selected requirement elicitation methods (self-composed)}
\end{longtable}
