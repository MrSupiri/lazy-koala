\section{The Novelty of the Research}

After a literature survey author came conclusion finding a root cause of any failure within a distributed system is a very difficult issue due to it not having single output we can try to predict and most researchers have built their own simulation of a distributed system by themselves since there isn't any open dataset about monitoring data mainly because it could contain sensitive information. 

Most currently established research is done towards creating statistical models like clustering and linear regression. Even though these algorithms perform very well in small-scale systems, they struggle to keep up when the monitoring data become very noisy with scale. Another problem none of these papers properly  addressed was constant changes to services. Most published research considers target services as static but in reality, these services can change even many times per day \citep{GoingtoM51:online}. So during this, the author is trying to utilize a recent trend in unsupervised learning which uses a convolutional autoencoder to find the abnormal pattern in time series data.

Finally, when it comes to aggregating service metrics the author is utilizing a fairly new technique called \ac{ebpf} which works by talking to the underlying kernel of the operating system and trying to spy the target service by intersecting low-level system calls.