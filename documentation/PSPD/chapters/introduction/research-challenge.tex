\section{Research Challenge \& Potential}

% Even though this project seems very straightforward and easy to implement from a high level, it becomes problematic when it comes to reaching targets the author has set for himself. For an example, interpretability was one of the most requested features from the industry experts and a must-have trait for mission-critical systems \citep{ribeiro2016should}. But it was left out of the project scope due to its complexity especially when it comes to an undergraduate level project. Other than that following are a few of the more difficult challenges the author is expected to face while conducting the research.

\subsection{Research Challenge}

\begin{itemize}[leftmargin=*] 
    \item \textbf{Highly seasonal and noisy patterns} - Monitoring metrics on microservices on production tends to have very unpredictable patterns depending on the traffic that has been sent to the service. The amount of traffic sent will depend on several external factors that are hard to determine. Modeling both temporal dependencies and static interdependencies found in telemetry data of services into a single graph will be very difficult and require a lot of fine-tuning and data engineering skills.
    \item \textbf{Overhead} - Modern deep learning models can solve any problem if we could give them an unlimited amount of data and processing power. Although in this case, the models need to optimize for efficiency over accuracy since having a monitoring system that consumes a lot more resources than the actual target system isn't effective.
    \item \textbf{Fit into Kubernetes eco-system} - Kubernetes has become the de-facto standard to managing distributed systems \citep{WhatisCo78:online}. So the author is planning to create a Kubernetes extension that will bridge the connection between monitored service and monitoring model. But Kubernetes itself has a very steep learning curve, even the original developers themselves have admitted, Kubernetes is too hard and complex for beginners \citep{Googlead4:online}.
    \item \textbf{Extraction of Telemetry} - Even though it's considered the best practice to implement telemetry exporting methods in the development phase of any application, developers often skip this part to save time. Sometimes it's required to depend on external applications that are developed by third parties which don't have means of exporting telemetry. When building an end-to-end root course indication platform, it's required to take these kinds of scenarios into account as well.
\end{itemize}

\subsection{Research Potential}

Initial feedback received for this project has been very positive due to the fact that this is a very common yet still unsolved issue in reliability engineering. Since this project is developed as a set of loosely coupled components, some of the experts expressed their interest in using individual components to solve some of the other problems they have been experiencing over time. Finally, this project can be used as a starting-off position for future researches which are focusing on specific areas of \ac{aiops} by replacing individual components of this with their owns.