\section{Project Management Methodology}

To manage task of this project authors decide to use \textbf{Agile PRINCE2}. Agile PRINCE2 built upon waterfall method which works best for projects with fixed deadlines and requirements with the added benefit of having regulated inputs and outputs \citep{WhatAreT79:online}.

\subsection{Deliverables}
\setlength\LTleft{0mm}
\begin{longtable}{|p{115mm}|p{35mm}|}
\hline
\textbf{Deliverable} & 
    \textbf{Date} \\ \hline
    \textbf{Draft Project Proposal} & 
    \multirow{2}{*}{02nd September 2021} \\
    A draft version of this proposal & 
     \\ \hline

    \textbf{A working beta of MicroSim}\label{microsim} & 
    \multirow{2}{*}{15th September 2021} \\
    MicroSim is a tool that simulates a distributed system within a Kubernetes cluster. & 
     \\ \hline

    % \textbf{Research Paper about MircoSim} & 
    % \multirow{2}{*}{16th October 2021} \\
    % MicroSim could have various other use-cases and could help in the development of this research domain. So the author is planning to release it as an open-source project with paper   so future research and benefits from this. & 
    %  \\ \hline

    \textbf{Literature Review Document} & 
    \multirow{2}{*}{21st October 2021} \\
    The Document explaining all the existing tools and published researches on the domain. & 
     \\ \hline

    \textbf{Project Proposal} & 
    \multirow{2}{*}{04th November 2021} \\
    The final version of this project proposal. & 
     \\ \hline

    \textbf{Software Requirement Specification} & 
    \multirow{2}{*}{25th November 2021} \\
    The Document all the key requirements that are gonna get address with this research. & 
     \\ \hline

    \textbf{Proof of Concept} & 
    \multirow{2}{*}{06th December 2021} \\
    Unoptimized prototype with all the main features working. & 
     \\ \hline

    \textbf{Interim Progress Report (IPR)} & 
    \multirow{2}{*}{27th January 2022} \\
    The document explaining all the preliminary findings and the current state of the project. & 
     \\ \hline

    \textbf{Project Specifications Design and Prototype (PSDP)} & 
    \multirow{2}{*}{03rd March 2022} \\ 
    A report detailing the initial design and implementation along with a working prototype. & 
     \\ \hline

    % \textbf{Test and Evaluation Report} & 
    % \multirow{2}{*}{17th April 2022} \\
    % A document with results of the project and conclusion made from those tests. & 
    %  \\ \hline

    % \textbf{Final Research Paper} & 
    % \multirow{2}{*}{14th April 2022} \\
    % A paper with results about this project & 
    %  \\ \hline

    \textbf{Final Project Report} & 
    \multirow{2}{*}{5th May 2022} \\
    Finalize version of the thesis. & 
     \\ \hline

\caption{Deliverables and due dates (self-composed)}
\end{longtable}

\subsection{Schedule}
Gantt chart is a visualization of the task with their respective timelines. Refer Appendix \ref{appendix:gantt-chart} to find the gantt chart for this project.



\subsection{Resource Requirement}

\subsubsection{Software Requirements}

\begin{itemize}[noitemsep,nolistsep] 
\item \textbf{Ubuntu / Arch Linux} - Since this project will use \ac{ebpf} as a dependency it will require a Linux kernel based operating system.
\item \textbf{Python / R} - Since this project has a data science components, relying on a language with a good data science eco-system will help to make the development easier.
\item \textbf{GoLang / Rust} - While GoLang has official client library made by Kubernetes developers themselves, kube-community has developed an excellent alternative in Rust.
\item \textbf{K3d / Minikube} - To create a Kubernetes cluster locally for development and testing.
\item \textbf{Intellij / VS Code} - An IDE provides tools streamline the development process.
\item \textbf{Latexmk / Overleaf} - Creating the project documation declaratively will helps to keep the formating consistent. Latex based editors provides this functionality.
\item \textbf{Google Drive / Github} - Offsite location to backup the codebase and related documents.
\item \textbf{ClickUp / Notion} - To manage the project and keep track of things to be done.
\end{itemize}

\subsubsection{Hardware Requirements}
\begin{itemize}[noitemsep,nolistsep] 
    \item \textbf{Quad-core CPU with AVX support} - AVX is a CPU instruction set which is optimze for vector operations. Having an AVX supported CPU could reduce the model inference time.
    \item \textbf{GPU with CUDA support and 2GB or more VRAM} - Both Tensorflow and Pytorch depend on CUDA for hardware-accelerated training.
    %  Training on GPU could save a lot of time increases the number of trial and error iterations that could be done.
    \item \textbf{16 GB or more Memory} - Running a microservices simulation locally will consume a lot of memory and while testing models will get loaded into RAM.
    \item \textbf{At least 40GB disk space} - To store the dataset, models docker containers while developing the project.
\end{itemize}

\subsubsection{Skill Requirements}
\begin{itemize}[noitemsep,nolistsep] 
    \item \textbf{Experience working with Kubernetes} - The author will be developing a Kubernetes extension so they need to know the inner workings of Kubernetes.
    \item \textbf{Data engineering} -  Developing a data encoding technique requires a lot of knowledge in how to manipulate a given dataset.
    \item \textbf{Model engineering} - Creating model from ground up is difficult task. So the author needs to have an in-depth idea about a machine learning framework and how different layers in the model work in order to fit them properly. 
\end{itemize}

\subsubsection{Data Requirements}
\begin{itemize}[noitemsep,nolistsep] 
\item \textbf{Monitoring dataset} -  This dataset can be collected using \hyperref[microsim]{MicroSim} tool author plan to develop to simulate distributed system.
\end{itemize}

\subsection{Risk Management}


\begin{longtable}{|p{4.8cm}|p{1.35cm}|p{1.8cm}|p{7cm}|}
    \hline
    \textbf{Risk Item} & 
    \textbf{Severity} & 
    \textbf{Frequency} & 
    \textbf{Mitigation Plan}
    \\ \hline
    
    The hypothesis the research is based on is wrong & 
    5 & 
    1 & 
    Present the findings and explain why the hypothesis was wrong.
    \\ \hline
    
    Failure in work computer & 
    4 & 
    3 & 
    Create regular off-site backups.
    \\ \hline
    
    Lack of domain knowledge & 
    2 & 
    3 & 
    Talk to a domain expert, keep research.
    \\ \hline
    
    Models not generalizing & 
    3 & 
    4 & 
    Explore different methods, Try cleaning up the dataset more.
    \\ \hline
    
    Dataset quality is not up to the standard & 
    4 & 
    1 & 
    Use a method used in related researches to create a new dataset.
    \\ \hline
    
    Running out of time & 
    1 & 
    2 & 
    Following a thorough work schedule.
    \\ \hline
    
    Getting sick and unable to work for few days & 
    3 & 
    3 & 
    Keeping few days of a buffer period before deadlines.
    \\ \hline
    \caption{Risks and mitigations (self-composed)}
\end{longtable}