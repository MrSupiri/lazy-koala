\section{Development Methodology}

Even though this project has a few clearly defined requirements, designing and developing them will require an iterative model as there is not a single best way to develop this and the author will be experimenting with different techniques. Thus the author decides on using \textbf{prototyping} as the Systems development life cycle (SDLC) Model for this project.\\

\subsection{Design Methodology}

To design the system diagrams for this project Structured System Analysis and Design Modelling (SSADM) methods will be used. SSADM make it easier to design the system iteratively and this complement the choice SDLC method Prototyping.

\subsection{Evaluation Methodology}

Since this is a software framework only performance and functional evaluation can be performed. To evaluate instrumentation components memory and CPU scalability will be used. The Data Science component of this research relies on a conventional autoencoder which is tasked with the sole purpose of recreating the input to the best of its ability. Due to this nature, only one metric that can be used to evaluate the model is training and testing loss functions \citep{gondara2016medical}. 

\subsection{Requirements Elicitation}

As the results of this project will be mostly used by \acp{sres} and system administrator the author is hoping to talk with few of the experts in the respective fields to get a better idea on what are the things to be expected from a system like this. Moreover as mentioned in \ref{sec:out-scope} this system is not designed to entirely replace existing monitoring systems, So the author is hoping to research about production monitoring systems and their workflows to understand how the proposed system could seamlessly integrate them. 
