\section{Achievement of Research Aims \& Objectives}

\subsection{Aim of the Project}

\section{Research Aim}

\textit{The aim of this research is to design, develop and evaluate a low overhead Kubernetes framework to collect, store and process telemetry data using deep learning to help system operators detect anomalies earlier in order to reduce the \ac{mttr} when the system is experiencing an anomaly.}

The aim of the project was successfully achieved by designing, developing, and evaluating a flexible framework which made the components interchangeable. This makes it easy for the data scientist to create reliable machine learning models to find the root causes of anomalies, which in turn help the system operators to reduce the \ac{mttr}.


\subsection{Research Objectives}

\begin{longtable}{|p{40mm}|p{91mm}|p{20mm}|}
    \hline
    \textbf{Research Objective} & \textbf{Justification} & \textbf{Status} \\ \hline
    Problem identification & A project with proper scope was identified to research on. & Completed \\ \hline
    Literature review & A detailed literature survey was conducted on all related topics. & Completed \\ \hline
    Developing an evaluation framework & \href{https://github.com/MrSupiri/MicroSim}{MicroSim} was developed and published. & Completed \\ \hline
    Data gathering and analysis & \ac{ebpf} agent \ac{gazer} and data preprocessor \ac{sherlock} was developed. & Completed \\ \hline
    Developing encoding methods & \ac{sherlock} uses the RGB colour scheme to represent the telemetry data. & Completed \\ \hline
    Developing the model & An autoencoder was developed to detect anomalies. & Completed \\ \hline
    Testing and evaluation & Both functional and non-functional tests were carried out to verify the capabilities of the project. Along with third party evaluation. & Completed \\ \hline
    Integration & \ac{lazy-koala-operator} was built to bind up all the components together. & Completed \\ \hline
    \caption{Achievements of Research Objectives(self-composed)}
  \end{longtable}
  
  