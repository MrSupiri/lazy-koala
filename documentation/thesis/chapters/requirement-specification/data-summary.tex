\section{Summary of Findings}

\begin{longtable}{|p{120mm}|p{3mm}|p{3mm}|p{3mm}|p{3mm}|p{3mm}|}
\hline
    \textbf{Findings} &
    \rotatebox{90}{\textbf{Literature Review  }} &
    \rotatebox{90}{\textbf{Interviews}} &
    \rotatebox{90}{\textbf{Self-evaluation}} &
    \rotatebox{90}{\textbf{Brainstorming}} &
    \rotatebox{90}{\textbf{Prototyping}} \\ \hline

    Finding the root cause of a problem during an outage is a time consuming task. &
    &
    X &
    X &
    &  \\ \hline

    It is possible to use machine learning to find anomalies from time series data. &
    X &
    &
    &
    & X \\ \hline

    Lot of companies are looking to migrate from monolithic architecture to microservices. &
    X &
    X &
    &
    &  \\ \hline

    There is a big trend towards “automating boring tasks”. &
      &
    X &
    X &
    &  \\ \hline

    Kubernetes is the most popular way to manage distributed systems. &
    X &
    X &
    X &
    &  \\ \hline

    \ac{ebpf} provides a low overhead method to collect telemetry data without additional instrumentation. &
    X &
    &
    X &
    & X \\ \hline

    Autoencoder are good at finding and forecasting patterns in data. &
    X &
    &
    & X 
    & X \\ \hline

    There aren't any established methods to test monitoring systems, Creating a testing toolkit for that would help future researchers. &
    X &
    X &
    &
    &  \\ \hline

    Having a dashboard which could show a blast radius of a system failure would help to reduce the \ac{mttr}. &
    & X 
    &
    X &
    &  \\ \hline

    The Kubernetes operator framework is the best way to build Kubernetes native applications. &
    &
    X &
    &
    X &
    X \\ \hline
    
    \caption{Summary of findings (self-composed)}
\end{longtable}