\section{Requirements Specifications}

Since this project touched the deep ends of both \ac{sre} and data science expectations. The prioritizes of the system must be managed to achieve a reliable and functioning system at end of the deadline. So to achieve this MoSCoW prioritization model was used.

\begin{longtable}{|p{25mm}|p{128mm}|}
\hline
    \textbf{Priority Level} &
    \textbf{Description} \\ \hline
    
    Must have  &
    Requirements which need to be met in order to have minimum viable product. \\ \hline

    Should have &
    Requirements that need to be completed to have a usable product. \\ \hline

    Could have &
    Nice to have requirements that would improve the quality of life of the system. \\ \hline
    
    Will not have &
    Requirements that will not get covered during this iteration but might get implemented in the future.\\ \hline
\caption{Requirement priorities (self-composed)}
\end{longtable}

\begin{longtable}{|p{35mm}|p{118mm}|}
\hline
    \textbf{Use Case ID} & \textbf{Use Case Name }              \\ \hline
    UC-01                & Deploy Lazy Koala                    \\ \hline
    UC-02                & Update Configuration                 \\ \hline
    UC-03                & Purge Lazy Koala                     \\ \hline
    UC-04                & Check for Root Courses               \\ \hline
    UC-05                & Generate Report                      \\ \hline
    UC-06                & Read from the database               \\ \hline
    UC-07                & Extract telemetry (Every 5 second)   \\ \hline
    % UC-08                & Update Network topology              \\ \hline
    UC-08                & Check for Anomalies (Every 1 minute) \\ \hline
    UC-09                & Write to the database                \\ \hline
    UC-10                & Reconcile on modified resources      \\ \hline
    UC-11                & Update cluster state                 \\ \hline
\caption{Use cases of the system (self-composed)}
\end{longtable}

% \newpage

\newcommand{\functionalRequirement}[5]{
    #1 &
    % \makecell[{{p{109mm}}}]{\textbf{#2}\\#3} &
    \textbf{#2} \newline #3 &
    #4 &
    #5  \\ \hline
}

\subsection{Functional Requirements}

\begin{longtable}{|p{9mm}|p{109mm}|p{14mm}|p{13mm}|}
\hline
    \textbf{ID} &
    \textbf{Requirement and Description} &
    \textbf{Priority Level} &
    \textbf{Use case} \\ \hline


    \functionalRequirement
    {FR01}
    {Users should be able to deploy the \ac{lazy-koala-operator} to an existing Kubernetes cluster.}
    {\ac{lazy-koala-operator} should work on any Linux-based Kubernetes cluster with version 1.22 or higher without any additional configuration from the user's end.}
    {Must have}
    {UC-01}
    
    \functionalRequirement
    {FR02}
    {Users should be able to remove the \ac{lazy-koala-operator} completely from the cluster.}
    {Once uninstalled all the provisioned resources should be cleaned up by the \ac{lazy-koala-operator} itself. }
    {Should have}
    {UC-03}
    
        
    \functionalRequirement
    {FR03}
    {Users should be able to specify which services need to be monitored.}
    {System should allow the user to exclude some services getting tracked.}
    {Must have}
    {UC-02}
    
        
    \functionalRequirement
    {FR04}
    {Users should be able to see the services monitored by \ac{lazy-koala-operator}}
    {System should be transparent to the user about monitored and unmonitored service.}
    {Could have}
    {UC-05}
    
        
    \functionalRequirement
    {FR05}
    {\ac{lazy-koala-operator} should deploy an instance of \ac{gazer} to every node in the cluster.}
    {In the Kubernetes cluster every node has a separate instance of the Linux kernel. So for every instance of Linux kernel, an instance of \ac{gazer} must be present to ensure all the relevant data is captured.}
    {Must have}
    {UC-01, UC-10}
    
        
    \functionalRequirement
    {FR06}
    {\ac{gazer} should intersect all “inet\_sock\_set\_state” kernel calls and export the relevant data to Prometheus.}
    {Whenever a userspace application makes a TCP call, this kernel method is invoked to communicate with the network interface. Inspecting the data structures of this will allow us to extract a lot of information about each TCP calls.}
    {Must have}
    {UC-07, UC-09}
    
        
    \functionalRequirement
    {FR07}
    {\ac{gazer} periodically poll the size of “sk\_ack\_backlog” for interested ports to export the relevant data to Prometheus.}
    {This kernel data structure holds the TCP connections that are left to be acknowledged. Knowing the size of this queue will help to understand the efficiency of each service.}
    {Should have}
    {UC-07, UC-09}
    
        
    \functionalRequirement
    {FR08}
    {\ac{gazer} should poll Kubernetes metric server periodically and export the relevant data to Prometheus.}
    {Sudden changes in CPU and Memory usage will be a good indication for an anomaly. So exporting those to be processed later will be wise.}
    {Must have}
    {UC-07, UC-09}
    
        
    \functionalRequirement
    {FR09}
    {\ac{lazy-koala-operator} should periodically check for changes in monitored services and update the \ac{gazer} ConfigMap.}
    {Kubelet is watching over all the services on the system and restarts them if they become unhealthy. With that, the IP address of that service is gonna change and \ac{lazy-koala-operator} is responsible to let the \ac{gazer} know of such changes.}
    {Should have}
    {UC-10, UC-11}
    
        
    \functionalRequirement
    {FR10}
    {\ac{gazer} should react to config updates in realtime.}
    {When a \ac{lazy-koala-operator} pushes a new config, \ac{gazer} should look for the new IPs without requiring a complete reset. Which is time consuming and expensive.}
    {Could have}
    {UC-10, UC-11}
    
        
    \functionalRequirement
    {FR11}
    {\ac{sherlock} should provision the list of models given by \ac{lazy-koala-operator}}
    {\ac{sherlock} should react to updates from \ac{lazy-koala-operator} and provision models on demand.}
    {Must have}
    {UC-10}
    
        
    \functionalRequirement
    {FR12}
    {\ac{sherlock} should periodically calculate the anomaly score for each of the monitored services and export it to Prometheus.}
    {Anomaly score is used by the UI to understand the spread of an anomaly.}
    {Must have}
    {UC-08, UC-09}
    
        
    \functionalRequirement
    {FR13}
    {\ac{lazy-koala-operator} should have a Web UI to visualize the service topology.}
    {UI should help users to visualize the spread of an anomaly throughout all of the monitored service.}
    {Should have}
    {UC-04, UC-05, UC-06}
    
        
    \functionalRequirement
    {FR14}
    {\ac{lazy-koala-operator} should add a finalizer for each of the provisioned resources.}
    {Finalizers ensure the parent of a resource won’t be deleted before all the children are cleaned up. This avoids leaving the cluster with orphaned resources that won’t be cleaned up without user intervention.}
    {Should have}
    {UC-03}

    \functionalRequirement
    {FR15}
    {\ac{lazy-koala-operator} should periodically fine-tune models.}
    {Microservices could get complex and change over time. To combat this \ac{lazy-koala-operator} make sure the models get adapted to the new changes done in monitored services.}
    {Will not have}
    {UC-04, UC-10}
    
    
    

\caption{Functional requirements (self-composed)}
\end{longtable}

\subsection{Non-Functional Requirements}

\begin{longtable}{|p{13mm}|p{89mm}|p{26mm}|p{18mm}|}
\hline
    \textbf{ID} &
    \textbf{Description} &
    \textbf{Specification} &
    \textbf{Priority Level} \\ \hline
    
    NFR1 &
    \ac{lazy-koala-operator} should follow Principle of Least Privilege when accessing Kubernetes APIs. &
    Security &
    Must have \\ \hline
    
    NFR2 &
    Systems should have fragmented architecture so each component can be individually scaled in order to save resources. &
    Scalability &
    Must have \\ \hline
    
    NFR3 &
    Each component should work individually such that users can install parts of the system they are interested in. &
    Usability &
    Could have \\ \hline
    
    NFR4 &
    \ac{gazer} should be limited for using only 100 mCPUs and 80MB of memory. &
    Performance &
    Should have \\ \hline
    
    NFR5 &
    \ac{sherlock} should be limited to using only 100 mCPUs and 100MB of memory. &
    Performance &
    Could have \\ \hline
    
    NFR6 &
    \ac{lazy-koala-operator} should be packaged as a Helm Chart for ease of use. &
    Usability &
    Must have \\ \hline
    
    NFR7 &
    Reconstruction error of \ac{sherlock} should be under 0.1\% &
    Performance &
    Could have \\ \hline
    
    NFR8 &
    \ac{lazy-koala-operator}’s reconciliation loop should use exponential backoff technique when there is an error while reconciling for a config change. &
    Reliability &
    Must have \\ \hline
    
    NFR9 &
    Follow Coding best practices and rely on linters for code formatting. &
    Maintainability &
    Could have \\ \hline
    
    NFR10 &
    The project should be backed by an automated CI/CD tool to test and build each component with each release. &
    Maintainability &
    Could have \\ \hline

\caption{Non-Functional requirements (self-composed)}
\end{longtable}