\section{Existing Work}

\subsection{Anomaly detection}

\begin{longtable}{| p{20mm} | p{43mm} | p{43mm} | p{43mm} |}
\hline
  \textbf{Citation} &
  \textbf{Technology summary} &
  \textbf{Improvements} &
  \textbf{Limitations} \\ \hline


  \cite{du2018anomaly} &
  Tested most of common machine learning methods to detect anomalies and benchmarked them &
  \vspace{-8mm}
  \begin{itemize}[leftmargin=*,noitemsep,nolistsep] 
    \item Used SLIs to monitored data
    \item A lot of good metrics (input data)
    \item Performance monitoring of services and containers
  \vspace{-7mm}
  \end{itemize} &
  \vspace{-8mm}
  \begin{itemize}[leftmargin=*,noitemsep,nolistsep] 
    \item Only be able to identify predetermined issues
    \item Require a sidecar that includes a lot of overheads
    \item Won't work with event-driven architectures (this is where most of the new systems are headed)
    \item Uses Supervised learning and it is nearly impossible to find real-world data with labels
  \vspace{-7mm}
  \end{itemize} \\ \hline


  \cite{kumarage2018anomaly} &
  The authors here propose a semi-supervised technique using a Variational Autoencoder to predict future time steps and calculate the difference between predicted and actual to detect anomalies. &
  \vspace{-8mm}
  \begin{itemize}[leftmargin=*,noitemsep,nolistsep] 
    \item Due to the difficulty of finding labelled research data, they settled on using a semi-supervised technique.
    \item Randomized decision trees used were used to select the most suitable characteristics for each component.
  \vspace{-7mm}
  \end{itemize} &
  \vspace{-8mm}
  \begin{itemize}[leftmargin=*,noitemsep,nolistsep] 
    \item The model will not be effortlessly transformable for other systems
    \item If more new key features were added to the system it will require a total retraining
  \vspace{-7mm}
  \end{itemize} \\ \hline


  \cite{kumarage2019generative} &
  Uses a bidirectional GAN to predict future timesteps and uses MSE between prediction and real to determine the anomalies &
  Experimented using a GAN to detect anomalies rather than using conventional autoencoders &
  \vspace{-8mm}
  \begin{itemize}[leftmargin=*,noitemsep,nolistsep] 
    \item Accuracy is around 60\%, which is unimpressive for use in production with mission-critical systems.
    \item As this is a GAN-based system, it may take numerous resources to run with production systems.
  \vspace{-7mm}
  \end{itemize} \\ \hline


  \caption{Comparison of anomaly detection methods in distributed systems (self-composed)}
\end{longtable}

\subsection{Root cause identification}

\begin{longtable}{| p{20mm} | p{40mm} | p{43mm} | p{46mm} |}
\hline
  \textbf{Citation} &
  \textbf{Technology summary} &
  \textbf{Improvements} &
  \textbf{Limitations} \\ \hline


  \cite{gonzalez2017root} &
  Detect failures in networks, using machine learning to generate knowledge graphs on historical data &
  \vspace{-8mm}
  \begin{itemize}[leftmargin=*,noitemsep,nolistsep] 
    \item Predictable system
    \item Automatic identification of dependencies between system events
    % \item independent of Domain experts
    \item Generalized to various systems
  \vspace{-7mm}
  \end{itemize} &
  \vspace{-8mm}
  \begin{itemize}[leftmargin=*,noitemsep,nolistsep] 
    \item Limited to network issues
    \item \acp{sres} must manually identify the issues, although the knowledge graph helped visualisation.
  \vspace{-7mm}
  \end{itemize} \\ \hline


  \cite{chigurupati2017root} &
  Proposed a way to detect hardware failures in servers using a probabilistic graphical model that concisely describes the relationship between many random variables and their conditional independence &
  \vspace{-8mm}
  \begin{itemize}[leftmargin=*,noitemsep,nolistsep] 
    \item Find hidden meaning in values that seems random
    \item Used a probabilistic approach to understand the relationship between inputs and outputs more clearly
    \item Gives all the possible root causes for a given problem
  \vspace{-7mm}
  \end{itemize} &
  \vspace{-8mm}
  \begin{itemize}[leftmargin=*,noitemsep,nolistsep] 
    \item Limited to hardware issues
    \item Require support from domain experts
    \item Can't account for unforeseen errors
  \vspace{-7mm}
  \end{itemize} \\ \hline


  \cite{wu2020microrca} &
  Find Performance bottlenecks in distributed systems using an attribute graph to find anomaly propagation across services and machines &
  \vspace{-8mm}
  \begin{itemize}[leftmargin=*,noitemsep,nolistsep] 
    \item Created a custom Fault Injection module
    \item Uses an attribute graph to localise to faulty service
    \item Application-agnostic by using a service mesh
    \item Rely on the service mesh to determine network topology
    \item Uses unsupervised learning
  \vspace{-7mm}
  \end{itemize} &
  \vspace{-8mm}
  \begin{itemize}[leftmargin=*,noitemsep,nolistsep] 
    \item Only able to identify 3 types of issues
    \item Checks only for performance anomalies
    \item Use the slow response time of a microservice as the definition of an anomaly
    \item Service meshes add a lot of overhead to systems
    \item Required direct connection between services
  \vspace{-7mm}
  \end{itemize} \\ \hline


  \cite{samir2019dla} &
  This detects and locates the anomalous behavior of microservices based on the observed response time using a HHMM &
  \vspace{-8mm}
  \begin{itemize}[leftmargin=*,noitemsep,nolistsep] 
    \item Custom HHMM model
    \item Self-healing mechanism
    \item Focus on performance detection and identification at the container, node, and service level
  \vspace{-7mm}
  \end{itemize} &
  \vspace{-8mm}
  \begin{itemize}[leftmargin=*,noitemsep,nolistsep] 
    \item The input data set scale is limited
    \item Require a sidecar
    \item Needs to predetermined thresholds
  \vspace{-7mm}
  \end{itemize} \\ \hline


  \caption{Comparison of root cause identification methods in distributed systems (self-composed)}
\end{longtable}

\subsection{Commercial products}

\begin{longtable}{| p{40mm} | p{55mm} | p{55mm} |}
\hline
  \textbf{Name} &
  \textbf{Futures} &
  \textbf{Limitations} \\ \hline


  Applied Intelligence by New Relic &
  \vspace{-8mm}
  \begin{itemize}[leftmargin=*,noitemsep,nolistsep] 
    \item Metric forecasting.
    \item Anomaly detection.
    \item Alert grouping to reduce noise.
  \vspace{-7mm}
  \end{itemize} &
  \vspace{-8mm}
  \begin{itemize}[leftmargin=*,noitemsep,nolistsep] 
    \item Lack of explanation for certain classifications.
    \item All the telemetry data need to be sent to a third party.
  \vspace{-7mm}
  \end{itemize} \\ \hline


  Watchdog by Datadog &
  \vspace{-8mm}
  \begin{itemize}[leftmargin=*,noitemsep,nolistsep] 
    \item Monitor the metric data of the entire system from the background.
    \item Monitor the log data.
    \item Highlight relevant components affected by an issue.
  \vspace{-7mm}
  \end{itemize} &
  \vspace{-8mm}
  \begin{itemize}[leftmargin=*,noitemsep,nolistsep] 
    \item Announced in 2018 but is still in private beta.
    \item Require code changes and tight integration with the datadog platform.
    \item Available demos about the system seem to be designed for demonstration purposes.
  \vspace{-7mm}
  \end{itemize} \\ \hline


  \caption{Comparison of commercial products for root cause analysis (self-composed)}
\end{longtable}