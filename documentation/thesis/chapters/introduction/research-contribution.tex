\section{Research Contribution}


\subsection{Domain Contribution}

With this research, the author first tried to develop a \textbf{cloud-native solution to create a configurable microservices system}, so this research and future research will have a standard environment to develop and evaluate their work. The author also hopes to build a lightweight and \textbf{low-overhead data collection pipeline} using \ac{ebpf} to collect telemetry of target services without any instrumentation from the user.

\subsection{Knowledge Contribution}

One of the main problems with monitoring microservices systems is that different services can be developed with different programming languages and frameworks, and those can contain different levels of noise\label{need-for-encoding}. Therefore, it is complicated for a single model to detect anomalies in any service, since some frameworks tend to use more resources while idle than others. To address this author is trying to come up with an \textbf{encoding method} so the model can be trained to monitor one framework and those learning will still be valid for another framework. With these encoded data, the author is hoping to develop a \textbf{convolutional autoencoder that will use unsupervised learning to spot out anomalies in a given data stream}. This may have better performance while using fewer resources convolutional layers are typically lightweight and good at pattern recognition \citep{oord2016wavenet}. Finally, the author plans to aggregate these predictions from the models into a pre-generated service graph and weigh them at \textbf{find all possible root causes}.
