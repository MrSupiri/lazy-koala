{\let\clearpage\relax\chapter{Use Case Descriptions}\label{appendix:use-case-description}}

\UseCaseDescription
{UC-01}
{Deploy Lazy Koala}
{Install \ac{lazy-koala-operator} to a Kubernetes cluster}
{Reliability Engineer}
{\begin{CompactItemizes}
    \item A Kubernetes cluster running.
    \item kubectl installed and configured to talk to the cluster.
    \item Helm CLI installed.
\end{CompactItemizes}}
{N/A}
{N/A}
{\begin{CompactEnumerate}
    \item Add Helm remote.
    \item Run helm install command.
    \item Kube API acknowledges the changes.
    \item Display content of Notes.txt
\end{CompactEnumerate}}
{{\begin{CompactEnumerate}
    \item Apply Kubernetes Manifest found in the code repository.
    \item Kube API acknowledges the changes.
\end{CompactEnumerate}}
{\textbf{E1}: A \ac{lazy-koala-operator} couldn’t achieve desired state
\vspace{-4mm}\begin{enumerate}
    \item The \ac{lazy-koala-operator} retries to achieve the desired state with an exponential backoff
\vspace{-7mm}\end{enumerate}}
{\begin{CompactItemizes}
    \item \ac{lazy-koala-operator} deployed on the cluster.
    \item Instance of \ac{gazer} deployed on every node.
    \item New permission rules are registered with Kube API.
\end{CompactItemizes}}}

\vspace{-2em}
\UseCaseDescription
{UC-02}
{Update Configuration}
{Add or Remove a service from a monitored list.}
{Reliability Engineer}
{\begin{CompactItemizes}
    \item kubectl installed and configured to talk to a Kubernetes cluster.
    \item The Kubernetes cluster has a \ac{lazy-koala-operator} deployed.
    \item Established port forwarding connection with \ac{lazy-koala-operator}.
\end{CompactItemizes}}
{N/A}
{N/A}
{\begin{CompactEnumerate}
    \item Visit the forwarded port on the local machine.
    \item Open the “Services” tab.
    \item Click Attach Inspector.
    \item Select the namespace and the service.
    \item Click Attach.
    \item Status update sent to kube API.
\end{CompactEnumerate}}
{{\begin{CompactEnumerate}
    \item Visit the forwarded port on the local machine.
    \item Open the “Services” tab.
    \item Scroll to the relevant record.
    \item Press the delete button next to the name.
    \item Confirm the action.
    \item Status update sent to kube API.
\end{CompactEnumerate}}
{\textbf{E1}: Kube API not available
\vspace{-4mm}\begin{enumerate}
    \item Show an error to the user asking to retry in a bit.
\vspace{-7mm}\end{enumerate}}
{\begin{CompactItemizes}
    \item A new Inspector resource is attached to the service.
\end{CompactItemizes}}}

\vspace{-2em}
\UseCaseDescription
{UC-03}
{Purge Lazy Koala}
{Remove Lazy Koala from a Kubernetes cluster.}
{Reliability Engineer}
{\begin{CompactItemizes}
    \item kubectl installed and configured to talk to a Kubernetes cluster.
    \item The Kubernetes cluster has a \ac{lazy-koala-operator} deployed.
\end{CompactItemizes}}
{N/A}
{N/A}
{\begin{CompactEnumerate}
    \item Find the helm release name.
    \item Run helm uninstall [release name].
\end{CompactEnumerate}}
{{\begin{CompactEnumerate}
    \item Locate Kubernetes Manifest found in the code repository.
    \item Run kubectl delete -f [manifest-file]
\end{CompactEnumerate}}
{N/A}
{\begin{CompactItemizes}
    \item All the resources provisioned by Lazy Koala including the \ac{lazy-koala-operator} itself get removed from the cluster.
\end{CompactItemizes}}}

\vspace{-2em}
\UseCaseDescription
{UC-07}
{Extract telemetry}
{Every 5 second \ac{gazer} will scrape the metric server}
{System Timer}
{\begin{CompactItemizes}
    \item \ac{gazer} is deployed to the cluster
\end{CompactItemizes}}
{N/A}
{Write to the database}
{\begin{CompactEnumerate}
    \item poll\_kube\_api function get invoked.
    \item \ac{gazer} looks at the config file and finds out the service it’s responsible for.
    \item Query metric server for each of the service names.
    \item Store it in local memory.
\end{CompactEnumerate}}
{{N/A}
{\textbf{E1}: metric server returns non 200 status code.
\vspace{-4mm}\begin{enumerate}
    \item Retry in the next iteration.
\vspace{-7mm}\end{enumerate}}
{\begin{CompactItemizes}
    \item Updated local memory with recent telemetry data.
\end{CompactItemizes}}}

\vspace{-2em}
\UseCaseDescription
{UC-09}
{Check for Anomalies}
{Check for Anomalies in each of the service}
{System Timer}
{\begin{CompactItemizes}
    \item An Instance of \ac{sherlock} is deployed.
\end{CompactItemizes}}
{N/A}
{Write to the database}
{\begin{CompactEnumerate}
    \item check\_anomlies function invoked.
    \item Query the database for telemetry for about the last 5 minutes.
    \item Do a forward pass on the model.
    \item Calculate the reconstruction loss.
    \item Store it in local memory.
\end{CompactEnumerate}}
{{N/A}
{\textbf{E1}: Database is unreachable
\vspace{-4mm}\begin{enumerate}
    \item Retry in the next iteration.
\vspace{-7mm}\end{enumerate}}
{\begin{CompactItemizes}
    \item Updated local memory with current reconstruction loss.
\end{CompactItemizes}}}

\vspace{-2em}
\UseCaseDescription
{UC-11}
{Reconcile on modified resources}
{Whenever a resource owned by the \ac{lazy-koala-operator} gets modified, kubelet invokes the reconciliation loop on the \ac{lazy-koala-operator}.}
{Kubelet}
{\begin{CompactItemizes}
    \item \ac{lazy-koala-operator} is deployed
\end{CompactItemizes}}
{Read the cluster state}
{N/A}
{\begin{CompactEnumerate}
    \item Resources get modified.
    \item Kubelet invokes a reconciliation loop on the \ac{lazy-koala-operator}.
    \item Check if the change is interesting.
    \item Update children resources accordingly.
\end{CompactEnumerate}}
{{\begin{CompactEnumerate}
    \item Resources get modified.
    \item Kubelet invokes a reconciliation loop on the \ac{lazy-koala-operator}.
    \item Check if the change is interesting.
    \item Stop execution.
\end{CompactEnumerate}}
{\textbf{E1}: Error while reconciling
\vspace{-4mm}\begin{enumerate}
    \item Retry with exponential backoff.
\vspace{-7mm}\end{enumerate}}
{\begin{CompactItemizes}
    \item Cluster in the new desired state.
\end{CompactItemizes}}}