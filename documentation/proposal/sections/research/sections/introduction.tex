\chapter{Introduction}

% Cloud computing is at steady rise for past few years  due to its scalability and  ease of use. With this change, a new programming paradigm called cloud-native was born. Cloud-native applications are often developed as a set of stand-alone microservices \citep{dragoni2017microservices} yet could depend on each other to provide a unified experience. 

% This helps different teams to work on different services which increases the development velocity. This works well for medium to large companies but over time when this mesh of services could become very complicated to a point where it's very difficult for one person to understand the whole system. When the system consists of 1000s of individual services talking and depending on each other, the network layer of that system becomes chaotic \citep{Introduc54:online} and failure in a single point can create a ripple effect across the system. When something like that happens it's really difficult to zero in on the exact point of failure quickly. 

% In this document author will explain the problem that's getting tackled, why it needs to be solved and how the author is planing to solve the problem within upcoming months.

% This document describes the problem, research prospects and the course of action for the upcoming months of research. In line with this, proofs of the problem and prior attempts are also explored. Finally, the estimated timelines of the project and expected deliverables are discussed.

This document was made to provide the necessary context about one of the main pain points that arises when it comes to maintaining distributed systems and a course of actions that could be taken to reduce them. To do that author will first give a brief overview of the target domain and existing steps that have already been taken, then the author talks about shortcomings and improvements that can be made to them. Finally, the document will be concluded with how the author will approach the problem and try to solve it.